\ifdefined\english

    \section*{Projects}

    \noindent \textbf{Rustle} \textbar{} \textbf{BlockSec} \textbar{} \textbf{Independent Development (2021.10 -
    2023.03)} \hfill \textsl{C++, Python}

    \begin{itemize}
        \item Developed an LLVM-based static analysis tool for NEAR blockchain smart contracts developed in Rust,
              addressing the challenges of Rust's complex syntax and the inefficiency of manual auditing. The project
              is open source on GitHub~\footnote{\url{https://github.com/blocksecteam/rustle}}.
        \item Detected and reported several real-world and high-value vulnerabilities in projects such as
              \href{https://blog.staderlabs.com/stader-near-incident-report-08-16-2022-afe077ffd549}{Stader},
              \href{https://coinculture.com/au/business/skyward-finance-reportedly-suffers-3m-exploit-on-near-protocol/}{Skyward},
              \href{https://cryptopotato.com/defi-crisis-averted-near-protocols-rainbow-bridge-attacker-loses-2-5-eth/}{Rainbow
              Bridge}, and
              \href{https://blog.perp.fi/dissecting-the-mango-exploit-how-risk-is-mitigated-on-perp-v2-eccc91987c91}{Mango
              (Solana)}.
        \item Participated in the \href{https://devpost.com/software/rustle}{NEAR MetaBUILD} hackathon and won a
              \$20,000 USD prize; Received \$50,000 USD in development funding from the
              \href{https://near.foundation}{NEAR Foundation}.
    \end{itemize}

    \noindent \textbf{Aliyun-OSS-Rust-SDK} \textbar{} \textbf{Alibaba Cloud} \textbar{} \textbf{Independent Development
    (2024.06 - 2024.07)} \hfill \textsl{Rust}

    \begin{itemize}
        \item Customers can directly use the SDK in their Rust projects to interact with the Object Storage Service
              (OSS) server without knowledge of server-side interface implementations. A codegen framework was also
              developed, cutting code redundancy by 87\% while greatly improving SDK usability and scalability.
        \item Independently developed a prototype that includes basic API implementations and extended API feature
              modules. The codebase comprises 15K lines, with 94\% CodeCov, and complete documentation.
    \end{itemize}

    \noindent \textbf{Anshun} \textbar{} \textbf{BlockSec} \textbar{} \textbf{Member (2023.04 - 2023.12)} \hfill
    \textsl{Python}

    \begin{itemize}
        \item Developed a static analysis tool for Ethereum smart contracts written in Solidity, utilizing various
              static analysis methods.
        \item Built an on-chain alert system based on Anshun. Successfully provided multiple timely warnings,
              preventing the loss of over \$300,000 USD.
        \item Developed a static analysis tool based on Anshun in collaboration with the Web3 leading exchange
              \href{https://defillama.com/protocols/Dexes}{Uniswap}, to analyze the latest version of its contract
              extensions.
    \end{itemize}

    \ifdefined\qr

        \noindent \textbf{Intelligent Medical Platform} \textbar{} \textbf{Course} \textbar{} \textbf{Lead for Account
        Module (2021.03 - 2021.07)} \hfill \textsl{Python}

        \begin{itemize}
            \item Developed a complete backend using Go, responsible for account-related operations and provided
                  interfaces to upper-level modules.
            \item The project is open source on
                  GitHub~\footnote{\url{https://github.com/futuretech6/software-engineering-backend}}.
        \end{itemize}

    \fi

    \noindent \textbf{RV-Kernel}~\footnote{\url{https://github.com/futuretech6/RV-Kernel}} \textbar{} \textbf{Course
    Project} \textbar{} \textbf{Independent Development (2020.10 - 2021.01)} \hfill \textsl{C, RISC-V Asm}

    \begin{itemize}
        \item Implemented a Linux kernel from scratch based on RISC-V hardware.
    \end{itemize}

    \noindent \textbf{Watchpoint-Based Kernel Isolation} \textbar{} \textbf{Laboratory} \textbar{} \textbf{Member
    (2021.07 - 2021.09)} \hfill \textsl{C, AArch64 Asm}

    \begin{itemize}
        \item Collaborative project with Huawei aimed at enhancing HarmonyOS's kernel security.
        \item Leveraged the hardware mechanism in Arm architecture for fault isolation between drivers and the kernel.
    \end{itemize}

    \ifdefined\qr\else

        \noindent \textbf{C2SafeRust} \textbar{} \textbf{Laboratory} \textbar{} \textbf{Leader (2021.04 - 2021.06)}
        \hfill \textsl{C}

        \begin{itemize}
            \item Attempted to automatically rewrite C-written kernel drivers to Rust to address potential security
                  issues in C-based kernels. Ensured kernel safety using Rust's built-in security mechanisms.
            \item Implemented detection and potential fixes for Use-After-Free (UAF) and Double-Free vulnerabilities.
        \end{itemize}

    \fi

\else

    \section*{项目经历}

    \noindent \textbf{Rustle} \textbar{} \textbf{BlockSec} \textbar{} \textbf{独立开发 (2021.10 - 2023.03)} \hfill
    \textsl{C++, Python}

    \begin{itemize}
        \item 针对 Rust 语言编写的 NEAR 区块链智能合约开发的基于 LLVM 的静态分析工具,解决了 Rust 语言语法复杂、人工审计效率较低且容易漏检的痛点,项目于 GitHub~\footnote{\url{https://github.com/blocksecteam/rustle}} 开源
        \item 已检测出多个实际存在且受到攻击的高价值漏洞:
              \href{https://blog.staderlabs.com/stader-near-incident-report-08-16-2022-afe077ffd549}{Stader}、
              \href{https://coinculture.com/au/business/skyward-finance-reportedly-suffers-3m-exploit-on-near-protocol/}{Skyward}、
              \href{https://cryptopotato.com/defi-crisis-averted-near-protocols-rainbow-bridge-attacker-loses-2-5-eth/}{Rainbow Bridge}、
              \href{https://blog.perp.fi/dissecting-the-mango-exploit-how-risk-is-mitigated-on-perp-v2-eccc91987c91}{Mango (Solana)}
        \item 参与 \href{https://devpost.com/software/rustle}{NEAR MetaBUILD} 黑客松并获得 \$20,000 USD 奖金;获得 \href{https://near.foundation}{NEAR 基金会} \$50,000 USD 开发资金
    \end{itemize}

    \noindent \textbf{Aliyun-OSS-Rust-SDK} \textbar{} \textbf{阿里云} \textbar{} \textbf{独立开发 (2024.06 - 2024.07)} \hfill
    \textsl{Rust}

    \begin{itemize}
        \item 主要功能包括:签名计算与身份校验、访问控制、请求响应包解析、错误重试、文件传输管理、服务接口调用;另外,首创性地利用语言特性提供了一套 codegen 框架用于全流程的配置自定义,大幅提升了 SDK 的易用性和可扩展性,将接口代码冗余减少 87\%
        \item 独立开发原型,包含基础 API 实现及拓展 API 功能模块;代码量 15K,CodeCov 94\%,文档完整
    \end{itemize}

    \noindent \textbf{Anshun} \textbar{} \textbf{BlockSec} \textbar{} \textbf{项目成员 (2023.04 - 2023.12)} \hfill
    \textsl{Python}

    \begin{itemize}
        \item 针对 Solidity 语言编写的以太坊智能合约静态分析工具;使用了符号执行、控制流构建、污点分析、依赖计算、路径求解、栈模拟等多种静态分析方法
        \item 工具在 precision 和 recall 指标上领先学术界及工业界的已知同类型工具;基于该工具建立了链上合约漏洞预警系统,已多次成功预警并挽回了超过 \$300K USD 的金融资产
        \item 与 Web3 头部交易所 \href{https://defillama.com/protocols/Dexes}{Uniswap} (成交量、总锁仓量均为第一) 合作,基于该工具框架开发了新版 Uniswap 合约的漏洞检测工具
    \end{itemize}

    \ifdefined\qr

        \noindent \textbf{医疗图像识别分类器} \textbar{} \textbf{学院项目} \textbar{} \textbf{项目成员 (2019.10 - 2020.05)} \hfill
        \textsl{Python}

        \begin{itemize}
            \item 基于 YOLO 实现医疗图像分类器,用于辅助医生诊断病灶类型、细胞病变等,在浙江大学医学院提供的测试集上 F1 高达 0.95
            \item 获评浙江大学校级优秀 SQTP 项目
        \end{itemize}

        \noindent \textbf{图片伪造鉴定系统} \textbar{} \textbf{国投智能(美亚柏科)} \textbar{} \textbf{项目成员 (2020.07 - 2020.08)} \hfill
        \textsl{Python}

        \begin{itemize}
            \item 基于 Bayer 阵列的 CMOS 成像原理,针对电子取证过程中存在的可能的图像伪造进行识别,通过复现图形学算法识别出图像中存在的肉眼不易察觉的拼贴、剪切、拖移、消除等后期手段,且效果较好
            \item 功能已上线美亚柏科的电子取证平台
        \end{itemize}

    \fi

    \noindent \textbf{RV-Kernel}~\footnote{\url{https://github.com/futuretech6/RV-Kernel}} \textbar{} \textbf{课程项目}
    \textbar{} \textbf{独立开发 (2020.10 - 2021.01)} \hfill \textsl{C, RISC-V Asm}

    \begin{itemize}
        \item 从零实现基于 RISC-V 硬件的 Linux 内核,包括内核启动模块、机器态特权态用户态切换、多进程调度器及调度算法、中断与异常机制、基于页表的 MMU 和缺页处理机制、基于 buddy system 和 slub allocator 的内存分配机制、基于 mmap 的内存映射机制及对应系统调用
    \end{itemize}

    \noindent \textbf{基于 Watchpoint 的 Linux 内核模块隔离} \textbar{} \textbf{实验室项目} \textbar{} \textbf{项目成员 (2021.07 -
    2021.09)} \hfill \textsl{C, AArch64 Asm}

    \begin{itemize}
        \item 与华为 2012 可信实验室的合作项目,用于提升 HarmonyOS 的内核安全性
        \item Arm 架构中存在 Watchpoint 硬件机制,是作为调试时的硬件断点,在 Load/Store 时被触发;项目旨在利用该机制设计能将驱动与内核进行故障隔离的内核保护机制
    \end{itemize}

    \ifdefined\qr\else

        % \noindent \textbf{智能医疗问诊平台}~\footnote{\url{https://github.com/futuretech6/software-engineering-backend}} \textbar{} \textbf{课程项目} \textbar{} \textbf{账号管理模块后端负责人 (2021.03 - 2021.07)} % \hfill Go, GORM, Echo, Jwt-Go, SwagGo

        % \begin{itemize}
        %     \item 使用 Go 编写一个完整的后端,账号管理模块负责与账户相关操作 (创建、登录、登出、状态保持、信息维护)并向上层模块提供接口
        % \end{itemize}

        \noindent \textbf{C2SafeRust} \textbar{} \textbf{实验室项目} \textbar{} \textbf{项目组长 (2021.04 - 2021.06)} \hfill
        \textsl{C}

        \begin{itemize}
            \item 为解决目前基于 C 的 kernel 中可能存在的安全问题,项目尝试研究出一种自动化的方式将 C 写的内核驱动重写为 Rust 的,并通过 Rust 自带的安全机制来保证内核的安全
            \item 实现了对于 use-after-free 与 double-free 漏洞的检测与可能的修复
        \end{itemize}

    \fi

\fi
